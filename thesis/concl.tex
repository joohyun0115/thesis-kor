
\section{결론}
\label{sec:concl}
%We proposed and evaluated a novel concurrent update method, \LDU, to maximally
%improve the performance scalability of Linux kernel in many core systems. 
우리는 새로운 동시적 업데이트 방법인 LDU를 제안하였고, 평가하였고, 
이것은 매니코어 시스템에서 리눅스 커널의 성능 확장성을 향상시킨다.
%Such improvement was possible through eliminating the  synchronized time-stamp
%counters management overhead found in a previous well-known scheme, OpLog.
우리의 방법은 사전에 연구되어 잘 알려진 로그 기반 알고리즘인 OpLog의 동기화된 타임
 스템프 카운터의 관리에 따른 오버헤드를 제거 할 수 있다. 
%Our experiments using a Linux kernel with our \LDU implementation
%revealed that the proposed \LDU shows better performance up to 2.7 times
%stock Linux kernel on the 120 core machine.
우리의 LDU를 리눅스 커널에 구현하여 수행한 우리의 실험은 기존 리눅스 커널에 비해 120코어 
공유 메모리 기반 컴퓨팅 환경에서 2.7배까지 성능 향상을 이루었다. 
%The \LDU is implemented on to Linux kernel 4.5-rc6 and available as open-source
%from \url{https://github.com/manycore-ldu/ldu}.
우리는 이러한 LDU를 리눅스 커널 4.5-rc6에 구현하였으며, 본 결과물은
아래 사이트에서 오픈소스로 이용할 수 있다.
\begin{center}
\url{https://github.com/manycore-ldu/ldu}
\end{center}

\newpage
\section{향후 연구}
%While the proposed technique achieves significant improvement in performance
%scalability through eliminating time-sensitive logs, there still remain the
%data structures to consider to further improve the scalability (i.e., stack and
%queue data structures).
우리가 제안한 기술은 시간에 예민한 로그를 업데이트 순간 마다 지움으로
 상당한 성능에 대한 향상성을 얻었으나, 이것은 여전히 특정한 자료구조(예를 들어 스택, 큐)
  같은 경우에는 남은 명령어들 마다 시간 순서가 필요하므로 적용하지 못한다.
%Future direction of research is to create a new synchronization scheme by
%combining two techniques(the \LDU and the OpLog) to support the stack and
%queue.
향후 연구로는 LDU의 기술과 OpLog의 기술을 통합하여 스택과 큐를 지원학 위해
 새로운 동기화 기술을 개발하는 것이다. 

%\section{Acknowledgments}
%This work was supported by Institute for Information \& communications
% Technology Promotion (IITP) grant funded by the Korea government (MSIP) (14-824-09-
%011, “Research Project on High Performance and Scalable Manycore Operating
% System”)

