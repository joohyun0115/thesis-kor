\section{Conclusion and Discussion}
\label{sec:concl}
We proposed and evaluated a novel concurrent update method, \LDU, 
to maximally improve the performance scalability of Linux kernel 
in many core systems. 
Such improvement was possible through eliminating the 
synchronized time-stamp counters management overhead 
found in a previous well-known scheme, OpLog.
Our experiments using a Linux kernel with our \LDU implementation
revealed that the proposed \LDU shows better performance up to 2.7 times
stock Linux kernel on the 120 core machine.

While the proposed technique achieves significant improvement
in performance scalability through eliminating time-sensitive
logs, there still remain the data structures to consider to further
improve the scalability (i.e., stack and queue data structures).
Future direction of research is to create a new synchronization scheme by combining 
two techniques(the \LDU and the OpLog) to support the stack and queue.

The \LDU is implemented on to Linux kernel 4.5-rc6 and available as open-source
from \url{https://github.com/manycore-ldu/ldu}.
%\section{Acknowledgments}
%This work was supported by Institute for Information \& communications
% Technology Promotion (IITP) grant funded by the Korea government (MSIP) (14-824-09-
%011, “Research Project on High Performance and Scalable Manycore Operating
% System”)

