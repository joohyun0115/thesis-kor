
\section{결론}

우리는 새로운 동시적 업데이트 방법인 LDU를 개발하였고, 
이것을 실제 매니코어 시스템에 적용하여 성능 확장성을 보였다.
우리의 방법은 사전에 연구된 로그 기반 알고리즘인 OpLog의 동기화된 타임스템프 카운터를 제거 할 수 있다. 
LDU를 최신 리눅스 커널에 구현하였고, 실험 결과 기존 리눅스 커널에 비해 2.7배까지 성능 향상을 이루었다. 

이처럼 동기화된 타임스탬프 카운터를 제거함과 동시에, 캐시 커뮤니케이션 병목 현상을 줄인
LDU는 기존 로그 기반 알고리즘들의 장점들을 모두 포함할 뿐만 아니라 추가적인 장점을 가진다.
첫째, 업데이트가 수행하는 시점 즉 로그를 저장하는 순간에는 Fine-grained 락이 필요가 없다.
따라서 락의 오버헤드 없이 동시적 업데이트를 수행할 수 있다
둘째로, 저장된 업데이트 명령어 로그를 Coarse-grained 락과 함께 하나의 코어에서 수행하기 때문에, 
캐시 효율성이 높아진다.
다음으로, 기존 여러 자료구조에 쉽게 적용할 수 있는 장점이 있다.
게다가 마지막으로, 로그를 저장하기 전에 로그를 쉽게 삭제하므로 더욱 빠르게 로그의 수를 줄일 수 있다. 

로그 기반의 방법을 사용하기 위해 LDU는 동기화된 타임스탬프 카운터를 사용하지 않고,
공유 메모리 시스템을 위한 새로운 LDU를 개발하였다.
LDU는 타임스탬프 카운터가 반드시 필요한 명령어는 업데이트 순간 제거하고 매번 로그를 
생성하지 않고 재활용하는 방법이다.
이로 인해 동기화된 타임스탬프 카운터의 현실적인 문제와 캐시-일관성 트래픽 때문에 발생하는 
병목 현상에 대한 문제를 동시에 해결하였다.
해결 방법은 분산 시스템에서 사용하는 로그 기반의 
동시 적 업데이트 방식과 최소한의 공유 메모리 시스템의 하드웨어 기반 동기화 기법을 조합하여 동시 적 
업데이트 문제를 해결하였다.

우리는 이러한 LDU를 리눅스 커널 4.5-rc6에 구현하였으며, 본 결과물은
아래 사이트에서 오픈소스로 이용할 수 있다.
\begin{center}
\url{https://github.com/manycore-ldu/ldu}
\end{center}

\newpage
\section{향후 연구}

우리가 제안한 기술은 로그 기반 방법 중 하나이다. 
이것은 아직 하드웨어 적으로 지원하지 않고, 소프트웨어들로도 검증되지 않은 
동기화된 타임 스탬프 카운터를 사용하지 않기 위해서, 
순서가 중요한 로그들을 업데이트 순간 마다 지우는 방법을 사용하였다. 

향후 연구로 접근 방법은 OpLog와 같이 타임 스탬프 기법을 사용할 수 있도록, 
여러 하드웨어 환경에서도 지원할 수 있는 소프트웨어 기반의 동기화된 타임 스탬프 카운터를 구현하는 것이다.
또한 우리의 방법은 업데이트 순간 락을 제거하고, 삭제 가능한 로그들을 제거함으로 
성능에 대한 향상 성을 얻었으나, 읽기 연산이 증가할 수 록 성능이 많이 떨어지는 것을 볼 수 있다.
그러므로 다른 향후 연구의 방향으로는 로그 기반 기술을 다른 동기화 기법과 통합하는 것이다. 
예를 들어 RCU와 같은 기법은 읽기 연산이 많을 수록 굉장히 높은 성능을 보여준다.
이처럼 RCU의 업데이트 부분을 수정하여, 로그 기반으로 작성하는 연구가 필요하다.
결론적으로 여러 동기화 기술을 통합한 새로운 동기화 기법을 개발하는 것이 필요하다.

