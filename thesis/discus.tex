\section{Discussion and future work}

%$$$$$$$$$$$$$$$$$$$$$$$$$$$$$$$$$$$$$$$$$$$$$$$$$$$$$$$$$$$$$$$$$$$$$$$$$$$$$$$$
%Paragraph 1: 더 범용적이게 만들어야 한다.
%$$$$$$$$$$$$$$$$$$$$$$$$$$$$$$$$$$$$$$$$$$$$$$$$$$$$$$$$$$$$$$$$$$$$$$$$$$$$$$$$
\ifkor
우리는 hight update rate data structure만 고려 하였지만, LDU는 보다 범용적인 상황에 적용 수 있게 만들어져야
한다.
그 이유는 실제 high update rate를 가진 상황은 자주 발생되는 상황이 아니다. 
따라서 data structure의 operation의 update rate이 적을 경우에는 불필요하게 log를 저장하므로 오히려 성능이 떨어
지는 역효과가 생긴다.
이것은 임베디드 시스템등 다양한 분야에 사용되는 리눅스 커널이 때문에 문제가 있다. 
따라서 항상 log-based로 처리하지 않고 update rate이 증가하면, 판단하여 log-based로 처리하도록 해야 한다.
즉 high update rate과 상관없이 좋은 성능을 보이는 범용적인 알고르즘으로 발전 시켜야한다.
\else
\fi


\ifkor
LDU의 두번 째 타입은 Per-core에 log를 저장하는 PLDU이다. 
PLDU의 원리는 기본적으로 log를 per-core 메모리에 저장하되 timestamp가 필요한 log만 shared memory
system의 atomic 연산으로 처리를 하자는 것이다.
When a process logs an insert operation on one core, migrates to another core,
and logs a remove operation, the remove should eventually execute after the
insert, so this condition needs the timestamp[].

즉 timestamp 없이 per-core로 merge된 로그가 아래와 같은 상황이 발생하면 문제가 발생한다. 

\inv{1}{A} \inv{2}{B} \inv{3}{C} \res{1}{A} \res{3}{C}

여기서 흥미로운 사실은 같은 object에 대해서 insert-remove operation을 update 할 때 수행해
주면 결국 per-core에 남은 operation은 insert 또는 remove operation만 남게 된다. 
즉 per-core log는 timestamp가 없어도 되는 operation만 남게 된다.
PLDU는 로그를 update-side absorbing 기술로 로그를 바로 바로 지워줌으로 timestamp를 없이도 per-core
log를 사용할 수 있도록 하였다.
\else
If a process logs an insert operation on one core,
migrates to another core, and logs a remove operation, the remove should
eventually execute after the insert[OpLog].
\fi


%$$$$$$$$$$$$$$$$$$$$$$$$$$$$$$$$$$$$$$$$$$$$$$$$$$$$$$$$$$$$$$$$$$$$$$$$$$$$$$$$
%Paragraph 2: reader 쪽을 개선해야 한다. 
%$$$$$$$$$$$$$$$$$$$$$$$$$$$$$$$$$$$$$$$$$$$$$$$$$$$$$$$$$$$$$$$$$$$$$$$$$$$$$$$$

\ifkor
현재 LDU는 data structure를 보호하기 위해 exclucive lock를 사용하도록 구현하였다. 
하지만 concurrent read를 위해 concurrent update와 concurrent read를 고려해서 구현해야 한다.
\else
\fi






%$$$$$$$$$$$$$$$$$$$$$$$$$$$$$$$$$$$$$$$$$$$$$$$$$$$$$$$$$$$$$$$$$$$$$$$$$$$$$$$$
%$$$$$$$$$$$$$$$$$$$$$$$$$$$$$$$$$$$$$$$$$$$$$$$$$$$$$$$$$$$$$$$$$$$$$$$$$$$$$$$$
%Reference Sentence 1
%$$$$$$$$$$$$$$$$$$$$$$$$$$$$$$$$$$$$$$$$$$$$$$$$$$$$$$$$$$$$$$$$$$$$$$$$$$$$$$$$





%$$$$$$$$$$$$$$$$$$$$$$$$$$$$$$$$$$$$$$$$$$$$$$$$$$$$$$$$$$$$$$$$$$$$$$$$$$$$$$$$
%Reference Sentence 2:LDU paper
%$$$$$$$$$$$$$$$$$$$$$$$$$$$$$$$$$$$$$$$$$$$$$$$$$$$$$$$$$$$$$$$$$$$$$$$$$$$$$$$$

