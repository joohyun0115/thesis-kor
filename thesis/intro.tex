\section{개요} \label{sec:intro}

%$$$$$$$$$$$$$$$$$$$$$$$$$$$$$$$$$$$$$$$$$$$$$$$$$$$$$$$$$$$$$$$$$$$$$$$$$$$$$$$$
%$$$$$$$$$$$$$$$$$$$$$$$$$$$$$$$$$$$$$$$$$$$$$$$$$$$$$$$$$$$$$$$$$$$$$$$$$$$$$$$$
%Background
%$$$$$$$$$$$$$$$$$$$$$$$$$$$$$$$$$$$$$$$$$$$$$$$$$$$$$$$$$$$$$$$$$$$$$$$$$$$$$$$$
%Achieving performance scalability has been a most important factor in 
%highly parallel systems with many cores (e.g., over 100 cores).
성능에 대한 확장성(Scalability)는 100코어 또는 1000 이상으로 구성된 
매니코어(Many-core) 시스템에서 가장 중요한 요소이다. 
%The performance scalability of a whole system is naturally limited by 
%scalability of underlying operating system kernel; Linux kernel has been 
%widely considered in ordinary systems.
전체 시스템의 성능 확장성은 일반적으로 운영체제의 커널(Kernel) 때문에 제한을 받는다.
이러한 운영체제 커널 중 가장 많이 사용되는 것은 리눅스(Linux) 커널이다. 
그 이유는 리눅스 커널은 멀티코어에 최적화가 되어 있기 때문이다.
%Previous research revealed that Linux kernel has significant problems limiting
%performance scalability in many core
% systems~\cite{SilasBoydWickizer2010LinuxScales48}~\cite{Changwoo2016UMSF} and the major sources of the problems are update lock contention in a few
%kernel data structures~\cite{mckenney2011parallel}~\cite{Matveev2015RLU}.
하지만 이렇게 멀티코어에 최적화된 리눅스 커널도 매니코어 시스템에서는 여전히 
성능에 대한 확장성에 문제가
있다~\cite{SilasBoydWickizer2010LinuxScales48}~\cite{Changwoo2016UMSF}.
확장성 문제중 가장 큰 문제는 커널의 자료구조 중 업데이트 락(Lock) 경쟁에 대한
문제 때문이다~\cite{mckenney2011parallel}~\cite{Matveev2015RLU}.



%$$$$$$$$$$$$$$$$$$$$$$$$$$$$$$$$$$$$$$$$$$$$$$$$$$$$$$$$$$$$$$$$$$$$$$$$$$$$$$$$
%$$$$$$$$$$$$$$$$$$$$$$$$$$$$$$$$$$$$$$$$$$$$$$$$$$$$$$$$$$$$$$$$$$$$$$$$$$$$$$$$
%Problem general
%$$$$$$$$$$$$$$$$$$$$$$$$$$$$$$$$$$$$$$$$$$$$$$$$$$$$$$$$$$$$$$$$$$$$$$$$$$$$$$$$
%Early research accomplishments regarding the update serialization problems
%include a number of concurrent
%update
% methods~\cite{Arbel2014ConcurrentRCU}~\cite{Matveev2015RLU}~\cite{Dodds2015SCT}.
이처럼 업데이트 직렬화(Serialization) 문제를 해결하기 위해 여러 동시적 업데이트(Concurrent Update)
 방법 들이 연구되고
 있다~\cite{Arbel2014ConcurrentRCU}~\cite{Matveev2015RLU}~\cite{Dodds2015SCT}.
%Such research provides bases to solve the update serialization problems, but
%does not effectively handle serious scalability bottleneck for update-heavy
% data structures.
동시적 업데이트 방법을 사용하여 업데이트 직렬화 문제를 해결하는 연구들은 
업데이트 비율에 따라 많은 성능 차이를 보인다.
이러하는 방법들은 높은 업데이트 비율을 가진 자료 구조 때문에 발생하는 확장성 문제에 
대해서는 여전히 효율적이지 않다.  
%Log-based algorithms~\cite{Hendler2010FC}~\cite{SilasBoydWickizerPth}
%have been proposed to solve this update serialization problem by reducing cache
% coherence-related overheads for update-heavy data structures.
높은 업데이트 비율을 가진 자료에 대한 해결책 중 하나는 캐쉬 통신 병목(cache communication bottleneck)
현상을 줄인 로그 기반(log-based) 알고리즘~\cite{Hendler2010FC}~\cite{SilasBoydWickizerPth}을
사용하는 것이다.
%When update operations occur, log-based algorithm logs the update
%operation and applies all operation logs to the data structure
%before read operation, so readers can read up to date data structure in a way
%similar to CoW(Copy On Write)~\cite{PaulDetailLWN}~\cite{Morrison2016SSM}.
로그 기반 알고리즘은 업데이트가 발생하면, 자료구조의 업데이트 명령(update operatin)을
퍼코어(per-core) 또는 원자적(atomic)하게 로그로 저장하고 읽기 명령(read operation)을 수행하기
 전에 저장된 로그를 수행하는것이다.
따라서, 리더(reader)는 최신 데이터를 읽게 되며, 이것은 마치 CoW(Copy On Write)와
유사하다~\cite{PaulDetailLWN}~\cite{Morrison2016SSM}.

%$$$$$$$$$$$$$$$$$$$$$$$$$$$$$$$$$$$$$$$$$$$$$$$$$$$$$$$$$$$$$$$$$$$$$$$$$$$$$$$$
%$$$$$$$$$$$$$$$$$$$$$$$$$$$$$$$$$$$$$$$$$$$$$$$$$$$$$$$$$$$$$$$$$$$$$$$$$$$$$$$$
%Problem 
%$$$$$$$$$$$$$$$$$$$$$$$$$$$$$$$$$$$$$$$$$$$$$$$$$$$$$$$$$$$$$$$$$$$$$$$$$$$$$$$$
%Among the log-based methods,
%S. Boyd-Wickizer \textit{et al.} proposed OpLog~\cite{SilasBoydWickizerPth},
% where each update operation generates a log with synchronized time-stamp counters
%and serialization of the logs based on the time-stamps solves the scalability
%bottleneck for update-heavy data structures: 
%the loggings are performed on to per-core memory instead of shared memory and
%thus eliminates cache communication overhead.
S. Boyd-Wickizer et al.는 동기화된 타임스탬프 카운터(synchronized timestamp counters) 기반의
퍼코어 로그를 활용하여 높은 업데이트 비율을 가진 자료구조를 대상으로 동시적 업데이트 문제를
해결함과 동시에 캐쉬 커뮤니케이션 병목현상(cache communication bottleneck)을
줄였다~\cite{SilasBoydWickizerPth}.
동기화된 타임스탬프 카운터 기반의 퍼코어 로그를 활용한 동시적 업데이트방법은
업데이트 부분만 고려했을 때, 퍼코어에 데이터를 저장함으로 굉장히 높은 성능 확장성을
 가진다[].
%However, the OpLog still has problems in scalability since the synchronized
% time-stamp counters necessitates time-stamp merging and ordering processes leading to 
%performance scalability problems in high CPU core counts.
하지만 퍼코어 기반의 동기화된 타임스탬프 카운터를 사용한 방법은 결국 타임스탬프 병합(timestamp merging)
과 정렬(ordering) 작업을 야기한다.
만약 코어 수가 늘어 날 경우, 로그를 자료 구조에 적용하는 과정에서 타임스탬프(timestamp)
 때문에 발생하는 추가적인 순차적 프로세싱sequential processing)이 요구된다.
이것은 결국 확장성과 성능을 저해한다. 


%$$$$$$$$$$$$$$$$$$$$$$$$$$$$$$$$$$$$$$$$$$$$$$$$$$$$$$$$$$$$$$$$$$$$$$$$$$$$$$$$
%$$$$$$$$$$$$$$$$$$$$$$$$$$$$$$$$$$$$$$$$$$$$$$$$$$$$$$$$$$$$$$$$$$$$$$$$$$$$$$$$
%Method
%$$$$$$$$$$$$$$$$$$$$$$$$$$$$$$$$$$$$$$$$$$$$$$$$$$$$$$$$$$$$$$$$$$$$$$$$$$$$$$$$
%
%We propose a novel lightweight log-based differed update method(\LDU) to
% achieve the maximum performance scalability for update-heavy data structures solving the 
%problems of sequential processing raised from the previous research:
%cache communication overhead in logging and time-stamp management cost during
% log ordering. 
%Such improved scalability could be achieved through combination of widely known 
%log-based concurrent update concept and our own way of efficient implementation
%methods of the log management scheme: log record slot reuse in the log queue
% and using minimal hardware-based synchronization
%method(compare and swap, test and set, atomic swap).
본 논문은 동기화된 타임스탬프 카운터를 이용함에 따라 생기는 추가적인 순차적 프로세싱(sequential processing) 문제를
해결하기 위해 공유 메모리 시스템을(shared memory system) 위한 새로운 LDU(Lightweitgh log-based
Deferred Update)를 개발하였다.
LDU는 타임스탬프 카운터가 필요한 명령어 로그(operation log)를 업데이트 순간 지우고,
 매번 로그를 생성하지 않고 재활용하는 방법이다.
이로인해 동기화된 타임스탬프 카운터 문제와 캐쉬 커뮤니케이션 병목현상에 대한 문제를 동시에 해결하였다.
해결 방법은 분산 시스템(distributed system)에서 사용하는 동기화된 타임스템프 로그 기반의 
동시적 업데이트 방식과 최소한의 공유 메모리 시스템의 하드웨어 기반 동기화(hardware-based
synchronization) 기법(compare and swap, test and set, atomic swap)을 조합하여 동시적 업데이트
문제를 해결하였다.

이처럼 동기화된 타임스탬프 카운터를 제거함과 동시에, 캐쉬 커뮤니케이션 병목 현상을 줄인
LDU는 기존 로그 기반 알고리즘들의 장점들을 모두 포함할 뿐만아니라 추가적인 장점을 가진다.
첫째로, 업데이트가 수행하는 시점 즉 로그를 저장하는 순간에는 fine-grain synchronization이 필요가 없다. 
따라서 동기화(synchronization) 오버헤드 없이 동시적 업데이트를 수행할 수 있다
둘째로, 저장된 업데이트 명령어 로그를 coarse-grained lock과 함께 하나의 코어에서 수행하기
때문에, 캐쉬 효율성이 높아진다~\cite{Hendler2010FC}.
다음으로, 기존 여러 자료구조에 쉽게 적용할 수 있는 장점이 있다.
게다가 마지막으로, 로그를 저장하기 전에 로그를 삭제하므로 보다 빠르게 로그의 수를 줄일 수 있다. 

%$$$$$$$$$$$$$$$$$$$$$$$$$$$$$$$$$$$$$$$$$$$$$$$$$$$$$$$$$$$$$$$$$$$$$$$$$$$$$$$$
%$$$$$$$$$$$$$$$$$$$$$$$$$$$$$$$$$$$$$$$$$$$$$$$$$$$$$$$$$$$$$$$$$$$$$$$$$$$$$$$$
%Result
%$$$$$$$$$$$$$$$$$$$$$$$$$$$$$$$$$$$$$$$$$$$$$$$$$$$$$$$$$$$$$$$$$$$$$$$$$$$$$$$$

%To evaluate our approach, we applied the \LDU to Linux kernel reverse
%page mappings(anonymous page mapping, file page mapping) that are considered as
%the major sources of limiting performance scalability due to their 
%update-heavy characteristics.
%We implemented the \LDU in a Linux 4.5-rc6.
%We evaluated the performance and scalability using a fork-intensive workload-
%AIM7~\cite{AIM7Benchmark}, Exim~\cite{Exim} from MOSBENCH~\cite{MOSBENCH}
%and Lmbench~\cite{mcvoy1996lmbench}-our design improves throughput and
% execution time on 120 core by 1.5x, 2.6x, 2.7x respectively, relative to stock Linux.
우리는 위와 같은 장점을 가지는 LDU를 리눅스 커널에서 높은 업데이트 비율 때문에 성능 확장성 
문제를 야기시키는 익명 역 매핑(anonymous reverse mapping)과 파일 역 매핑(file reverse mapping)에
적용하였다.
또한 우리는 LDU를 리눅스 커널 버전 4.5.rc4에 구현하였고, Fork가 많이 발생하는(fork-intensive) 워크로드인
AIM7~\cite{AIM7Benchmark}, MOSBENCH~\cite{MOSBENCH}의 Exim~\cite{Exim},
Lmbench~\cite{mcvoy1996lmbench}를 대상으로 성능 개선을 보였다.
개선은 기존 리눅스 커널에 비해 120코어에서 각각 1.5x, 2.6x, 2.7x 배 성능 향상을 가진다.

%$$$$$$$$$$$$$$$$$$$$$$$$$$$$$$$$$$$$$$$$$$$$$$$$$$$$$$$$$$$$$$$$$$$$$$$$$$$$$$$$
%$$$$$$$$$$$$$$$$$$$$$$$$$$$$$$$$$$$$$$$$$$$$$$$$$$$$$$$$$$$$$$$$$$$$$$$$$$$$$$$$
%Contribution 정리
%$$$$$$$$$$$$$$$$$$$$$$$$$$$$$$$$$$$$$$$$$$$$$$$$$$$$$$$$$$$$$$$$$$$$$$$$$$$$$$$$
\newpage
\section{논문의 기여}\label{sec:introcontri}
%\textbf{Contributions.} Our research makes the following contributions:
본 논문은 다음과 같은 기여를 하였다.
\begin{itemize}
%\item We have developed a novel lightweight log-based deferred update method
%eliminating the sources of limiting performance scalability in update-heavy
%% data structures with efficient log management implementation.
\item 우리는 높은 업데이트 비율를 가지는 자료구조를 위한 새로운 로그 기반 동시적 업데이트
 방법인 LDU를 개발하였다.
LDU는 동기화된 타임스탬프 카운터를 이용함에 따라 생기는 시간 정렬과 머징에 의한 추가적인
순차적 프로세싱 문제를 최소한의 하드웨어 동기화 기법을 사용하여 해결한 방법이다.
LDU는 하드웨어 동기화 기법을 이용하여 LDU는 로그를 업데이트 순간 지우고 로그를
재활용한다.
\item 
%We applied the \LDU in Linux kernel to two reverse mapping(anonymous, file) on
%an 120 core system to reduce fork scalability bottleneck.
%Our design improved throughput and execution time from 1.5x through 2.7x on 120
% core.
우리는 LDU을 현실적인(practical)한 매니코어 시스템인 인텔(Intel) 제온(XEON)
 120코어 위에 동작하는 리눅스 커널의 2가지 역 매핑 (익명, 파일)에 적용하여, 리눅스 fork
 성능 확장성 문제를 해결하였다.
Fork 관련 벤치마크 성능은 워크로드 특성에 따라 1.6x부터 2.2x까지 개선되었다.
\end{itemize}
