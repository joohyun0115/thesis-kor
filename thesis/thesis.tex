\documentclass[doctor,korean,final]{kmu}

% 본문 시작
\begin{document}

% 목차 (Table of Contents) 생성
\tableofcontents

% 표목차 (List of Tables) 생성
\listoftables

% 그림목차 (List of Figures) 생성
\listoffigures

% 위의 세 종류의 목차는 한꺼번에 다음 명령으로 생성할 수도 있습니다.
%\makecontents
%% 한글로 쓴 논문에는 본문에 영문 글자를 쓰지 않는다. 다만, 꼭 필요할 때에는 ‘한글 낱말 (영문 낱말)’ 꼴로 적는다.
%% 이하의 본문은 LaTeX 표준 클래스 report 양식에 준하여 작성하시면 됩니다.
%% 하지만 part는 사용하지 못하도록 제거하였으므로, chapter가 문서 내의
%% 최상위 분류 단위가 됩니다.
%% You cannot use 'part'

\chapter{머릿말}
본문을 한글로 작성할 때 머릿말로 시작을 하시는 게 좋습니다. \cite{FD1}
인용은 다음과 같이 합니다 \cite{RVP1}-\cite{ML2}.
인용은 뒤에 인용을 쓰는 칸이 있습니다. 참고하여서 인용하시길 바랍니다 \cite{SOCA2,EF2}.
한글 논문에는 영어를 쓰지 마시기 바랍니다. 


\chapter{본문 작성}

\section{작성}

장과 절 그리고 부절로 본문을 작성하실 수 있습니다.

\subsection{자동}

이것들은 자동으로 차례에 들어가게 됩니다.

\section{한글 논문}

한글 논문에는 영어를 쓰지 마시기 바랍니다.

\chapter{그림, 표}

\section{그림과 표를 본문에서 이야기하기}

본문에서 그림과 표에 관해 이야기를 할 때도 인용에서처럼 하시면 됩니다.

%%
%% 표 삽입 예시
%% Example. how to insert table
%%
\begin{table}[t]
\caption{표 제목을 넣으십시오.
}
\label{mag-tab1}
\begin{center}
\begin{tabular} {ccccccccccc}
\hline\hline
& & BF &\multicolumn{2}{c}{SW-I}&&\multicolumn{2}{c}{SW-II}&SW-III&CAP&\\
\cline{4-5} \cline{7-8}
&               &   &  Para & Ferro &&   Para &  Ferro &      &      &\\
\hline
& $E$ (eV)      & 0 & 7.796 & 7.832 && 10.418 & 10.408 & 11.5 & 13.2 &\\
& $M$ ($\mu_B$) & 0 &     0 &  1.94 &&      0 &   2.06 &    0 &    0 &\\
\hline\hline
\end{tabular}
\end{center}
\end{table}

%%
%% 그림 삽입 예시
%% Example. how to insert graph
%%
%% Note. 가급적 \includegraphics 명령을 사용하십시오.
%% Recommen : Use \includegraphics to insert graph.
%%
%\begin{figure}[t]
%    \centerline{\includegraphics[width=12.5cm]{sample-fig1}}
%    \caption{그림 제목을 넣으십시오.
%    } \label{mag-fig1}
%\end{figure}


\chapter{맺음말}

마지막은 맺음말로 하는 것을 권합니다.

%%
%% 참고문헌 시작
%% bibliography
%% 위는 보기이므로 학과 또는 논문의 특성에 맞게 조정 가능함. 다만, 참고문헌마다 충분한 정보가 들어 있어야 한다.
\begin{thebibliography}{00}
\bibitem{FD1} 박상우, \underline{동시 송수신 안테나를 두 개 쓰는 협력 인지 무선통신망에 알맞은 전 이중 통신},
한국과학기술원 석사 학위 논문, 2016.

\bibitem{RVP1} 송익호, 박철훈, 김광순, 박소령, \underline{확률변수와 확률과정}, 자유아카데미, 2014.

\bibitem{ML1} 송익호, 안태훈, 민황기, \underline{인지 무선에서의 광대역 주파수 검출 방법 및 장치}, 특허등록번호
10-1494966, 2015년 2월 12일.

\bibitem{SOCA1} 호우위시, 이원주, 이승원, 안태훈, 이선영, 민황기, 송익호, “선형 판별 분석에서 부류안 분산 행렬의 영 공간
재공식화,” \underline{한국통신학회 2012년도 추계종합학술발표회}, 대한민국 고려대학교, 242-243쪽, 2012년 11월.

\bibitem{EF1} 민황기, 안태훈, 이승원, 이성로, 송익호, “비간섭 전력 부하 감시용 고차 적률 특징을 갖는 전력 신호 인식,”
\underline{한국통신학회논문지}, 제39C권, 제7호, 608-614쪽, 2014년 7월.

\bibitem{FD2} S. Park, \textit{Full-Duplex Communication for Cooperative
Cognitive Radio Networks with Two Simultaneous Transmit and Receive Antennas},
Master Thesis, Korea Adv. Inst. Science, Techn., Daejeon, Republic of Korea,
2016.

\bibitem{RVP2}  I. Song, J. Bae, and S. Y. Kim, \textit{Advanced Theory of
Signal Detection: Weak Signal Detection in Generalized Observations},
Springer-Verlag, 2002.

\bibitem{ML2} I. Song, T. An, and J. Oh, \textit{Near ML decoding method based
on metric-first search and branch length threshold,} registration no. US
8018828 B2, Sep. 13, 2011, USA.

 \bibitem{SOCA2} H.-K. Min, T. An, S. Lee, and I. Song, “Non-intrusive appliance
load monitoring with feature extraction from higher order moments,” in
\textit{Proc. 6th IEEE Int. Conf. Service Oriented Computing, Appl.,} Kauai,
HI, USA, pp. 348-350, Dec. 2013.

\bibitem{EF2} I. Song and S. Lee, “Explicit formulae for product moments of
 multivariate Gaussian random variables,” \textit{Statistics, Probability
Lett.,} vol. 100, pp. 27-34, May 2015.

\end{thebibliography}


%%
%% 사사 시작
%% Acknowledgement
%% 사사 작성은 선택사항임
% @command acknowledgement 감사의글
% @options [1 | 2 | 3 |4 ]
% - 1 : 본문과 감사의 글이 둘 다 한글일 때  | 2 : 본문은 한글인데 감사의 글이 영어일 때 | 3 :  본문과 감사의 글이 둘 다 영어일 때  | 4 : 본문은 영어인데 감사의 글이 % 한글일 때 

\acknowledgment[1]
언제나 저를 바른 길로 이끌어 주시는 송익호 교수님께 큰 고마움을 느낍니다.
끝으로 오늘의 제가 있을 수 있도록 사랑으로 키워 주신 가족들에게 감사드립니다.
저의 이 작은 결실이 그분들께 조금이나마 보답이 되기를 바랍니다.


%% 본문 끝
\end{document}